\documentclass[11pt, oneside]{article}   	% use "amsart" instead of "article" for AMSLaTeX format
\usepackage{geometry}                		% See geometry.pdf to learn the layout options. There are lots.
\geometry{letterpaper}                   		% ... or a4paper or a5paper or ... 
%\geometry{landscape}                		% Activate for for rotated page geometry
%\usepackage[parfill]{parskip}    		% Activate to begin paragraphs with an empty line rather than an indent
\usepackage{graphicx}				% Use pdf, png, jpg, or eps� with pdflatex; use eps in DVI mode
								% TeX will automatically convert eps --> pdf in pdflatex		
\usepackage{amssymb}

\title{Brief Article}
\author{The Author}
%\date{}							% Activate to display a given date or no date

\begin{document}
\maketitle
%\section{}
%\subsection{}
\section{Name of the project}
HomeActivity: Recognizing activities in a home using sensor information. 

\section{Some description of the problem, and why it is interesting/challenging.}
In this problem, we tackle the problem of activity recognition using available sensor information. 
 
- Interesting:
	- Applications to healthcare; aberration to eldery mvoing; 
		- long term monitoring of degenerative heath;
	- Energy savings
		- 
	- Security
		- 
	- Intelligent Homes
		- 
- Challenging
	- Sensor datasets could be noisy
		- Doors open and close
		- May not be accurately labeled
		
	- mapping of multiple labels to a sensor activity
		- fridge ; make coffee and make cooking
		- 
	
	
Healthcare , security; energy efficiency; 


Discover human activity 

Sensor values identify the labels; 



User comfort versus near optimal schedule
 
1 month dataset; 25Feb to 1 March
Imbalanced class problem
where the total number of a class of data (positive) is far less than the total number of another class of data (negative)
machine learning algorithms and works best when the number of instances of each classes are roughly equal
\section{Methods that might be explored to solve it. }
HMM, 
CRF,
SVM
\section{ Timeline for finishing the project. }

\end{document}  