\section{Related Works}

The probabilistic models discussed in our project represent the state of the art models used in activity recognition. Tapia et al. used the naive Bayes model in combination with the raw feature representation on two real world datasets recorded using a wireless sensor network~\cite{emtapia}. HMMs were used in work by Patterson et al. and were applied to data obtained from a wearable RFID reader in a house where many objects are equipped with RFID tags~\cite{djpatt}. In work by van Kasteren et al. the performance of HMMs and CRFs in activity recognition was compared on a realworld dataset recorded using a wireless sensor network~\cite{tvkasteren}. Duong et al. compared the performance of HSMMs and HMMs in activity recognition using a laboratory setup in which four cameras captured the location of a person in a kitchen setup~\cite{tduong}. One type of model that we have not included in our comparison are hierarchical models. They have been successfully applied to activity recognition from video data~\cite{sluhr}, in an office environment using cameras, microphones and keyboard input~\cite{noliver} and on data obtained from a wearable sensing system ~\cite{asubr}.
The related works show part of the models in their work, respectively. It is hard to compare across different works. Instead, we implemented a set of models and ran the experiments on the same real word dataset. The evaluation results could be used as the baseline for future research. 



